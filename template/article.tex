\documentclass[12pt]{article}			% тип документа "статья"
\usepackage[utf8]{inputenc}				% поддержка UTF8
\usepackage[english,russian]{babel}  	% поддержка русского языка
\usepackage[unicode, pdftex]{hyperref}	% гиперссылки
\usepackage{outlines}					% многоуровневые списки
\usepackage{indentfirst}				% первый абзац
\usepackage{cmap}       				% теперь из pdf можно копипастить русский текст
\usepackage{geometry}

\setcounter{secnumdepth}{3}				% вложенность секций до третьего уровня
\geometry{verbose,a4paper,tmargin=2cm,bmargin=2cm,lmargin=2.5cm,rmargin=1.5cm} % отступы страницы

\pagestyle{empty} 	% нумерация страниц выкл. 
%\pagestyle{plain}	% нумерация страниц вкл.

\begin{document}

\begin{center}
	\Huge{\textbf{Название статьи}}
\end{center}

\hfill

\noindent
\textit{Здесь описывается тематика статьи и общие факты, возможно поднятие проблематики и её краткое историографическое описание.}

\begin{flushright}
	\today
\end{flushright}
\hfill

	
\section{Раздел первый}
Текст первого раздела. Текст первого раздела. Текст первого раздела. Текст первого раздела. Текст первого раздела. Текст первого раздела. Текст первого раздела. Текст первого раздела. Текст первого раздела. Текст первого раздела. Текст первого раздела. 
\begin{outline}[itemize]
	\1 элемент1
	\1 элемент2 \footnote{пример первой сноски}
\end{outline}

\section{Раздел второй}
Текст второго раздела. Текст второго раздела. Текст второго раздела. Текст второго раздела. Текст второго раздела. Текст второго раздела. Текст второго раздела. Текст второго раздела. Текст второго раздела. Текст второго раздела. Текст второго раздела. 
\begin{outline}[enumerate]
	\1 элемент1 \href{https://www.google.com/}{гиперссылка}
	\1 элемент2 \footnote{пример второй сноски}
	\1 элемент3
\end{outline}

\section{Заключение}
Текст заключения. Текст заключения. Текст заключения. Текст заключения. Текст заключения. Текст заключения. Текст заключения. Текст заключения. Текст заключения. Текст заключения. Текст заключения. 


\end{document}
\documentclass[12pt]{article}			% тип документа "статья"
\usepackage[utf8]{inputenc}				% поддержка UTF8
\usepackage[english,russian]{babel}  	% поддержка русского языка
\usepackage[unicode, pdftex]{hyperref}	% гиперссылки
\usepackage{outlines}					% многоуровневые списки
\usepackage{indentfirst}				% первый абзац
\usepackage{cmap}       				% теперь из pdf можно копипастить русский текст
\usepackage{geometry}

\setcounter{secnumdepth}{3}				% вложенность секций до третьего уровня
\geometry{verbose,a4paper,tmargin=2cm,bmargin=2cm,lmargin=2.5cm,rmargin=1.5cm} % отступы страницыыы

\pagestyle{empty} 	% нумерация страниц выкл. 
%\pagestyle{plain}	% нумерация страниц вкл.

\begin{document}

\begin{center}
	\Huge{\textbf{Память}}
\end{center}

\hfill

\noindent
\textit{В данной статье предпринята попытка обобщить прочитанный материал по теме улучшения памяти и запоминания информации. Также предлагаются общие рекомендации проверенные на практике. Функцию памяти нельзя рассматривать в отрыве от здоровья организма в целом, поэтому рекомендации которые здесь даны касаются в том числе и образа жизни в целом, а так же привычек.}

\begin{flushright}
	\today
\end{flushright}
\hfill
	
\section{Ухудшают память}
\begin{outline}[itemize]
	\1 регулярное недосыпание
	\1 курение и/или регулярное употребление алкоголя (даже в небольших количествах)
	\1 неправильное питание (<<пищевой мусор>>)
	\1 отсутствие регулярной аэробной активности
\end{outline}

\section{Улучшают память}
\begin{outline}[itemize]
	\1 полноценный сон, не менее 7 часов
	\1 отказ от вредных привычек
	\1 сбалансированное питание \footnote{отсутствие пищевого мусора в рационе, а так же уменьшение до минимума солёной, острой, жирной и жаренной пищи, так же отказ от мучного, сахаросодержащего и газированных напитков}
	\1 регулярная кардионагрузка \footnote{длительная быстрая ходьба или бег, в зависимости от общей массы тела и состояния опорно-двигательного аппарата. Такого рода нагрузку рекомендуется давать через день или 3 раза в неделю, постепенно доводя длительность до 40-60 минут}
	\1 разучивание стихов и чтение книг
	\1 изучение новой области знаний
\end{outline}

\section{Фармакология}
Препараты и БАДы способные оказать благотворное действие на функцию памяти:
\begin{outline}[itemize]
	\1 омега 3-6-9 \footnote{хорошим источником является ежедневный приём 1-2 столовых ложки льняного или оливкового масла}
	\1 витамин D3
	\1 витамины группы В (В6,В9 и В12 в особенности)
	\1 магний
	\1 адаптогены (родиола, женьшень, элеутерококк)
	\1 ноотропы (ноотропил, с осторожностью по рекомендации врача)
\end{outline}
	
\section{Общие рекомендации запоминания}
\begin{outline}[enumerate]
\1 \textbf{Метод мнемоники.} Например, чтобы запомнить последовательность цветов в спектре, мы используем фразу <<\textbf{К}аждый \textbf{О}хотник \textbf{Ж}елает \textbf{З}нать \textbf{Г}де \textbf{С}идит \textbf{Ф}азан>>. Первые буквы по названию цветов: красный, оранжевый, желтый, зелёный, голубой, синий, фиолетовый.

\1 \textbf{Заинтересованность.} Создайте фактор заинтересованности или найдите мотивацию. Информация, которая интересная усваивается лучше. Представьте, чем вам будет полезна данная информация.

\1 \textbf{Ассоциации.} Обращайтесь к ассоциациям (метод Цицерона). Суть его в том, что единицы информации, которые необходимо запомнить, мысленно расставляются в знакомой вам комнате, улице или другом хорошо знакомом пространстве в определенном порядке. Достаточно вспомнить знакомое пространство, чтобы вспомнить и воспроизвести информацию.

\1 \textbf{Логическая целостность.} При запоминании информации её нужно переработать в логически связанные между собой сущности/объекты, т.е. когда наличие одного объекта влечёт существование другого или когда из одного вытекает другое и т.д. Таким образом <<дёргая>> за объект или вспомнив о нём, можно легко вспомнить о связанном с ним объекте. Т.е. не должно быть такого, что какая-то часть запоминаемой информации никак не связана с остальной. Всегда можно придумать <<историю>>, которая бы связывала две каких-либо на первый взгляд не связанные сущности.

\1 \textbf{Пересказ.} Найдите <<свободные уши>> и расскажите собеседнику о прочитанном материале, предварительно, возможно, записав для себя план рассказа. Это позволит ещё лучше структурировать и <<упаковать>> информацию в своей голове, так она лучше запомнится/<<уляжется>>, а пояснения и ответы на появившиеся вопросы собеседника позволят более глубоко самому разобраться в излагаемой теме.

\1 \textbf{Повторения.} Последнее и самое важное это повторения, нейронные связи формируются именно в результате повторений. Этот факт был обнаружен Германом Эббингаузом ещё в 1885г. \href{https://ru.wikipedia.org/wiki/%D0%9A%D1%80%D0%B8%D0%B2%D0%B0%D1%8F_%D0%B7%D0%B0%D0%B1%D1%8B%D0%B2%D0%B0%D0%BD%D0%B8%D1%8F}{кривая Эббингауза}. Суть метода такова, что после прочтения информации, нужно сделать ряд повторений:

\begin{itemize}
	\item первое повторение -- \textbf{сразу} по окончании чтения
	\item второе повторение -- через \textbf{20—30 минут} после первого повторения
	\item третье повторение -- через \textbf{1 день} после второго
	\item четвёртое повторение -- через \textbf{2—3 недели} после третьего
	\item пятое повторение -- через \textbf{2—3 месяца} после четвёртого повторения
\end{itemize}

Для использования этого метода является необходимым писать \textbf{краткий конспект} с тезисами о прочитанном, увиденном или услышанном, чтобы можно было к нему вернуться спустя вышеописанные периоды времени и воспользоваться им для повторений.
\end{outline}


\end{document}
\documentclass[12pt]{article}			% тип документа "статья"
\usepackage[utf8]{inputenc}				% поддержка UTF8
\usepackage[english,russian]{babel}  	% поддержка русского языка
\usepackage[unicode, pdftex]{hyperref}	% гиперссылки
\usepackage{outlines}					% многоуровневые списки
\usepackage{indentfirst}				% первый абзац
\usepackage{cmap}       				% теперь из pdf можно копипастить русский текст
\usepackage{csquotes}		  			% пакет для цитат
\usepackage{geometry}

\setcounter{secnumdepth}{3}				% вложенность секций до третьего уровня
\geometry{verbose,a4paper,tmargin=2cm,bmargin=2cm,lmargin=2.5cm,rmargin=1.5cm} % отступы страницы

\pagestyle{empty} 	% нумерация страниц выкл. 
%\pagestyle{plain}	% нумерация страниц вкл.

\begin{document}
	
	\begin{center}
		\Huge{\textbf{Педагогика}}
	\end{center}
	
	\hfill
	
	\noindent
	\textit{В этой статье я попытаюсь собрать в единое целое сущетсвующие методики и подходы к образованию, проанализировать их, а так же попытаюсь сформулировать своё видение построения процесса обучения. Пока что данная статья содержит выдержки из прочитанного мною материала по данной теме более 7 лет назад.}
	
	\begin{flushright}
		\today
	\end{flushright}
	\hfill
	
	Ян Амос Коменский (28.03.1592-15.11.1670) — чешский педагог-гуманист, писатель, общественный деятель, епископ Чешскобратской церкви, основоположник научной педагогики, систематизатор и популяризатор классно-урочной системы.
	
	\begin{displayquote}
		\textit{
			\dq Ум без образования не более способен принести большой урожай, чем поле без обработки, каким бы оно ни было плодородным \dq
		} -- Цицерон
		
		\textit{
			\dq Пой учителю хвалу, до скончанья века, из тебя учитель твой сделал человека \dq
		} -- Восточная Мудрость
		
		\textit{
			\dq Обучать -- значит вдвойне учиться \dq
		} -- Жозеф Жубер
	\end{displayquote}
	
	
	\section{Методы обучения и оценки полученных знаний}
	
	{\bfТрадиционные методы обучения:}
	
	\begin{small}{\it(направленные на 95\%)}\end{small}
	\begin{itemize}
		\item лекция;
		\item работа с книгой;
		\item упражнения;
		\item лабораторная работа;
		\item беседа;
		\item рассказ;
	\end{itemize}
	
	\begin{small}{\it(направленные на 5\%)}\end{small}
	\begin{itemize}
		\item объяснение;
		\item учебная дискуссия;
		\item демонстрация;
		\item взаимообучение;
		\item практическая работа;
		\item самостоятельная работа.
	\end{itemize}
	
	
	{\bfСовременные методы обучения:}
	
	\begin{itemize}
		\item метод рефлекции;
		\item метод ротаций;
		\item \dqлидер\dq-\dqведомый\dq;
		\item обмен опытом.
	\end{itemize}
	
	{\bfМетоды воздействия на личность учащихся:}
	
	\begin{itemize}
		\item убеждение;
		\item упражнени и приучение (режимные/в полезной деятельности/специальные);
		\item обучение (получение новых знаний; формирование навыков; оценка и проверка полученных знаний/навыков);
		\item стимулирование.
	\end{itemize}
	
	{\bfМетоды оценки знаний:}
	
	\begin{itemize}
		\item устная проверка
		\item поурочный балл
		\item контрольная работа
		\item проверка домашних заданий
		\item рейтинговая система
		\item тестирование
	\end{itemize}
	
	{\bfО степени умственного развития говорят:}
	
	\begin{itemize}
		\item скорость усвоения материала
		\item темп усвоения материала
		\item количество обдумываний
		\item способность к самостоятельной систематизации и обобщению
	\end{itemize}
	
	{\bfТри кита интеллектуальной активности:}
	
	\begin{itemize}
		\item совершенствование памяти
		\item концентрация внимания
		\item усиление восприятия
	\end{itemize}
	
	\section{Педагогическая система Коменского}
	
	{\bfТри кита интеллектуальной активности:}
	
	\begin{itemize}
		\item детство (до 6);		[физический рост и развитие органов чувств]
		\item отрочество (7-12);	[развитие памяти и воображения)]
		\item юношество (13-18);	[развитие более высокого уровня мышления)]
		\item возмужание (19-24);	[развитие воли и способности к гармоничному существованию]
	\end{itemize}
	
	{\bfГлава 16. Общие требования обучения и учения, т.е. как учить и учиться.}
	
	{\itОсновоположение 1}
	Образования человека надо начинать в весну жизни, т.е. в детстве.
	Утренние часы для занятий наиболее удобны.
	Всё, подлежащее изучению, должно быть распределено сообразно ступеням возраста – так, чтобы предлагалось для изучения только то, что доступно восприятию в данном возрасте.
	
	{\itОсновоположение 2}
	Подготовка материала: книг и др. учебных пособий – заранее.
	Развивать ум ранее языка.
	Реальные учебные предметы предпосылать формальным.
	Примеры предпосылать правилам.
	
	{\itОсновоположение 4}
	В школах должен быть установлен порядок, при котором ученики в одно и то же время занимались бы только одним предметом.
	
	{\itОсновоположение 6}
	С самого начала юношам, которым нужно дать образования, следует дать основы общего образования (распределить учебный материал так, чтобы следующие затем занятия не вносили ничего нового, а представляли только некоторое развитие полученных знаний).
	Любой язык, любые науки должны быть сперва преподаны в простейших элементах, чтобы у учеников сложились общие понятия их как целого.
	
	{\itОсновоположение 7}
	Вся совокупность учебных занятий должна быть тщательно разделена на классы – так, чтобы предшествующее всегда открывало дорогу последующему и освещало ему путь.
	Время должно быть распределено с величайшей точностью – так, чтобы на каждый год, месяц, день и час приходилось своя особая работа.
	
	{\bfГлава 17. Основы лёгкости обучения и учения}
	
	{\itОсновоположение 1}
	Образование юношества должно начинаться рано.
	У одного и того же ученика по одному и тому же предмету должен быть только один учитель.
	По воле воспитателя прежде всего должны быть приведены в гармонию нравы.
	
	{\itОсновоположение 2}
	Всеми возможными способами нужно утверждать в детях горячее стремление к знанию и учению.
	Метод обучения должен уменьшать трудности учения, чтобы оно не возбуждало в учениках неудовольствия и не отвращало их от дальнейших занятий.
	
	{\itОсновоположение 3}
	Каждая наука должна быть заключена в самые сжатые, но точные правила.
	Каждое правило нужно излагать немногими, но самыми ясными словами.
	Каждое правило должно сопровождаться многочисленными примерами, чтобы стало очевидно, как разнообразно его применение.
	
	{\bfГлава 18 Основы прочности обучения и учения}
	Основательно должны рассматриваться только те вещи, которые могут принести пользу.
	Всё последующее должно опираться на предыдущее.
	Всё должно закрепляться постоянными упражнениями.
	Всё нужно изучать последовательно, сосредоточивая внимания на чём-то одном.
	На каждом предмете нужно останавливаться до тех пор, пока он не будет понят.
	
	{\bfГлава 26 О школьной дисциплине}
	«Школа без дисциплины есть мельница без воды»
	Для поддерживания дисциплины руководствоваться:
	Постоянными примерами воспитатель сам должен показывать пример.
	Наставлениями, увещеваниями, иногда и выговорами[9].
	
	
	{\bf9 правил искусства обучения наукам:}
	\begin{itemize}
		\item всему, что должно знать, нужно обучать;
		\item все, чему обучаешь, нужно преподносить учащимся, как вещь действительно существующую и приносящую определенную пользу;
		\item всему, чему обучаешь, нужно обучать прямо, а не окольными путями;
		\item всему, чему обучаешь, нужно обучать так, как оно есть и происходит, то есть путём изучения причинных связей;
		\item все, что подлежит изучению, пусть сперва предлагается в общем виде, а затем по частям;
		\item части вещи должно рассмотреть все, даже менее значительные, не пропуская ни одной, принимая во внимание порядок, положение и связь, в которой они находятся с другими частями;
		\item все нужно изучать последовательно, сосредоточивая внимание в каждый данный момент только на чем-либо одном;
		\item на каждом предмете нужно останавливаться до тех пор, пока он не будет понят;
		\item различия между вещами должно передавать хорошо, чтобы понимание всего было отчетливым.
	\end{itemize}
	
	
	{\bf16 правил искусства развивать нравственность:}
	\begin{itemize}
		\item добродетели должны быть внедряемы юношеству все без исключения;
		\item прежде всего основные, или, как их называют «кардинальные» добродетели: мудрость, умеренность, мужество и справедливость;
		\item мудрость юноши должны почерпать из хорошего наставления, изучая истинное различие вещей и их достоинство;
		\item умеренности пусть обучаются на протяжении всего времени обучения, привыкая соблюдать умеренность в пище и питье, сне и бодрственном состоянии, в работе и играх, в разговоре и молчании;
		\item мужеству пусть они учатся, преодолевая самих себя, сдерживая своё влечение к излишней беготне или игре вне или за пределами положенного времени, в обуздывании нетерпеливости, ропота, гнева;
		\item справедливости учатся, никого не оскорбляя, воздавая каждому своё, избегая лжи и обмана, проявляя исполнительность и любезность;
		\item особенно необходимые юношеству виды мужества: благородное прямодушие и выносливость в труде;
		\item благородное прямодушие достигается частым общением с благородными людьми и исполнением на их глазах всевозможных поручений;
		\item привычку к труду юноши приобретут в том случае, если постоянно будут заняты каким-либо серьёзным или занимательным делом;
		\item особенно необходимо внушить детям родственную справедливости добродетель— готовность услужить другим и охоту к этому;
		\item развитие добродетелей нужно начинать с самых юных лет, прежде чем порок овладеет душой;
		\item добродетелям учатся, постоянно осуществляя честное!;
		\item пусть постоянно сияют перед нами примеры порядочной жизни родителей, кормилиц, учителей, сотоварищей;
		\item однако нужно примеры сопровождать наставлениями и правилами жизни для того, чтобы исправлять, дополнять и укреплять подражание;
		\item самым тщательным образом нужно оберегать детей от сообщества испорченных людей, чтобы они не заразились от них;
		\item и так как едва ли удастся каким-либо образом быть настолько зоркими, чтобы к детям не могло проникнуть какое-либо зло, то для противодействия дурным нравам совершенно необходима дисциплина.
	\end{itemize}
	
\end{document}